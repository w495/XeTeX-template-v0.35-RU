%& -shell-escape

%%% Copyright (c) 2011, Илья w-495 Никитин
%%%
%%% Разрешается повторное распространение и использование
%%% как в виде исходного кода, так и в двоичной форме,
%%% если таковая будет получена, с изменениями или без, 
%%% при соблюдении следующих условий:
%%%
%%%     * При повторном распространении исходного кода 
%%%       должно оставаться указанное выше уведомление 
%%%       об авторском праве, этот список условий
%%%       и последующий отказ от гарантий.
%%%     * Ни имя w-495, ни имена друзей или консультантов
%%%       не могут быть использованы в качестве поддержки
%%%       или продвижения продуктов, основанных на этом коде 
%%%       без предварительного письменного разрешения. 
%%%
%%% Этот код предоставлен владельцом авторских прав 
%%% и/или другими сторонами <<как она есть>> 
%%% без какого-либо вида гарантий, выраженных явно 
%%% или подразумеваемых, включая, но не ограничиваясь ими, 
%%% (подразумеваемые) гарантии коммерческой ценности и пригодности 
%%% для конкретной цели. Ни в коем случае, если не требуется 
%%% соответствующим законом, или не установлено в устной форме, 
%%% ни один владелец авторских прав и ни одно другое лицо,
%%% которое может изменять и/или повторно распространять программу,
%%% как было сказано выше, не несёт ответственности,
%%% включая любые общие, случайные, специальные 
%%% или последовавшие убытки, вследствие использования 
%%% или невозможности использования программы 
%%% (включая, но не ограничиваясь потерей данных, 
%%% или данными, ставшими неправильными, или потерями
%%% принесенными из-за вас или третьих лиц, 
%%% или отказом программы работать совместно 
%%% с другими программами), даже если такой владелец или другое
%%% лицо были извещены о возможности таких убытков.
%%% 

%%% Документ нужно собирать только в XeLaTeX:
%%% 	$>xelatex имя-файла.tex
%%% Для этого должны быть установлены пакеты XeLaTeX и XeTeX
%%% 	в TeXLive или MikTeX или иной, 
%%% если она поддерживает последние обновдения CTAN.


%% Вариант для 14 pt
% \documentclass[unicode, 14pt, a4paper,oneside,fleqn]{extarticle}

%& -shell-escape

\documentclass[utf8x, 14pt, a4paper,oneside,fleqn]{extarticle}
% fleqn --- сдвигает формулы влево

%% Варианты {}:
% book
% report
% article
% letter
% minimal (???)

\usepackage{styles/init}


	% подключаем набор стилей 
	% там были определены базовые настройки шрифтов
	% и пакетов роботы с графикой и листингами
	
	% При не обходимости шрифты следует переопределить
	% потому что, если в Вашей системе 
	% не окажется нужных шрифтов, pdf не соберется
	
	% текущее положение включаемых файлов --- ./src

	\hypersetup{ 
		unicode=false,
		% %	pdffitwindow=false,
		% % pdfstartview={FitH}, % как отображать страницу {FitH}, {FitW}
		pdftitle={Это шаблонный документ XeTeX v0.35}, 
		pdfauthor={Илья w-495 Никитин},
		pdfcreator={XeTeX + TexMaker + w-495}, 
		pdfsubject={Тема}, 
		pdfproducer={w-495}, 
		pdfkeywords={Шаблон}
	}

\begin{document}
    \begin{onehalfspacing}


        %%%%%%%%%%%%%%%%%%%%%%%%%%%%%%%%%%%%%%%%%%%%%%%%%%%%%%%%%%%%%%%%%%%%%%%%%%%%%%%%
        %%%
        %%% бесполезное содержимое
        %%%

        \input{src/work-titlepage} 	    % титульный лист (1)
        \tableofcontents 		        % оглавление

        \pagebreak

        %%%%%%%%%%%%%%%%%%%%%%%%%%%%%%%%%%%%%%%%%%%%%%%%%%%%%%%%%%%%%%%%%%%%%%%%%%%%%%%%
        %%%
        %%% дополнительное (свое) задание верхнего колонтитула
        %%%
        %%%
        %	\makeatletter
        %	\renewcommand{\@oddhead}{ \textcolor{blue}{Лекция (задача) \arabic{lections}} \hfil \par
        %	\hfil  \leftmark \hfil \rightmark }
        %	\makeatother


        %%%%%%%%%%%%%%%%%%%%%%%%%%%%%%%%%%%%%%%%%%%%%%%%%%%%%%%%%%%%%%%%%%%%%%%%%%%%%%%%
        %%%
        %%% полезное содержимое
        %%%

        % пример %%%%%%%%%%%%%%%%%%%%%%%%%%%%%%%%%%%%%%%%%%%%%%%%%%%%%%%%%%%%%%%%%%%
        % это просто пример, который, якобы может показать основные особенности,
        % фичи и недостатки,

        
\Csection{Введение}

\index{шаблон}
Это шаблон написан мною для меня.\\
Изначально сделано для работы только в pdf\LaTeX только в \textbf{utf8}.
Сейчас шаблон используется исключительно для  \XeTeX.
Не для чего другого его на данный момент использовать нельзя.

Папки: \index{папки}
\begin{itemize}
	\item \textit{styles} --- стили и настройки документов.
	\item \textit{img} --- растровые картинки.
	\item \textit{src} --- исходные \TeX-исходники.
	\begin{itemize}
		\item Файлы, расположенные в папке \textit{/src/examples/}~---~примеры.
		\item Остальные файлы --- настоящие шаблоны.
		\begin{itemize}
			\item Префикс \textit{work-} --- относится к лабораторным или курсовым работам.
		\end{itemize}	
	\end{itemize}
\end{itemize}

\index{архитектура}

\pagebreak


        \supersection{Возможности}
        \section[Исходный код]{Исходный код}

%%%%%%%%%%%%%%%%%%%%%%%%%%%%%%%%%%%%%%%%%%%%%%%%%%%%%%%%%%%%%%%%%%%%%%%%%%%%%%%%
%%%
\subsection{lstlisting}
\index{lstlisting}
\index{листинги}
\index{исходники}
\index{код}
Исходный код с помощью пакета \textbf{listings} (или \textbf{listingsutf8}).
Пакет хорошо работает с однобайтовыми кодировками, но при любых настроках отказался дружить с utf8.

\begin{lstlisting}
    \usepackage[utf8]{inputenc}							% кодировка, тут очень аккуратно
\end{lstlisting}

\colorbox{yellow}{Проблема глобальна}.
И я не нашел стандартного пути решения (в pdf\LaTeX и \XeTeX~---~в $\Lambda$ ее нет).

%% \begin{lstlisting}[language=Tex, escapeinside='']
\begin{lstlisting}[escapeinside='', firstnumber=100]
    %\usepackage{listingsutf8}
    \usepackage{listings}
    \lstset{
        language=Tex,
        tabsize=2,
        breaklines,
        columns=fullflexible,
        flexiblecolumns,
        frame=tb ,
        numbers=left,
        numberstyle=\footnotesize\color{gray},
        escapechar = |, % 'можно вывалиться в \TeX'
        extendedchars = false,
            % extendedchars = true,
                %% да именно так но не  \true
                %% \true == false
        inputencoding = utf8, % кодировка, очень аккуратно тут
            % inputencoding = utf8/cp1251, % кодировка, очень аккуратно тут
        keepspaces = true,
        belowcaptionskip=5pt
    }
\end{lstlisting}

Пути решения:
\begin{itemize}
    \item Не использовать русских комментариев
    \item Использовать \textbf{verbatim},
\end{itemize}

\begin{lstlisting}[language=ConfigNetTopo]
[localhost]

[[7200]]
image = /usr//bin/Dynamips/images/c7200-is-mz.122-40.bin
    ram = 128
    npe = npe-300

[[3640]]
    image = /usr/bin/Dynamips/images/3640-is-mz.122-40.bin
    ram = 64
    model = 3640
    slot0 = NM-1E
    slot1 = NM-1FE-TX
    slot2 = NM-1FE-TX

[[ROUTER Alpha]]
model = 7200
    slot0 = C7200-IO-FE
    slot1 = PA-8E
    f0/0 = LAN 1
    e1/0 = Client09 e0/0
    e1/1 = Client10 e0/0
    console = 2000

[[ROUTER Client09]]
    model = 3640
    f1/0 = LAN 2
    f2/0 = LAN 29
    console = 2010

\end{lstlisting}


\pagebreak

%%%%%%%%%%%%%%%%%%%%%%%%%%%%%%%%%%%%%%%%%%%%%%%%%%%%%%%%%%%%%%%%%%%%%%%%%%%%%%%%
%%%
\subsection{verbatim}

\index{verbatim}
Его проблемы:
\begin{itemize}
    \item Нет подсветки синтаксиса
    \item Нет номеров строк
    \item Надо использовать пробелы вместо табуляции
\end{itemize}

\begin{verbatim}
    %\usepackage{listingsutf8}	%%  ---> %% utf8/cp1251
    \usepackage{listings}
    \lstset{
        language=Tex,
        tabsize=2,
        breaklines,
        columns=fullflexible,
        flexiblecolumns,
        frame=tb ,
        numbers=left,
        numberstyle={\footnotesize},
        extendedchars = false,
                % extendedchars = true,
                        %% да именно так но не  \true
                        %% \true == false
        inputencoding = utf8, % кодировка, очень аккуратно тут
                % inputencoding = utf8/cp1251,
        belowcaptionskip=5pt
    }
\end{verbatim}

\pagebreak %% Разрыв страницы :-)

        
\section{Алгоритмы и псевдокод}


\subsection{clrscode, codebox}

\begin{codebox}
	\Procname{$\proc{\tt Ничего не делает}$}
	\li \For 
			$i \gets 0 $ \To $\infty$
	\li \Do $i \gets i$
\end{codebox}

\index{сортировка вставкой}
\begin{codebox}
	\Procname{$\proc{\tt \textcolor{red}{Сортировка методом вставки} }(A)$}
	\li \For $j \gets 2$ \To $\id{length}[A]$
	\li \Do
		$\id{key} \gets A[j]$
	\li \Comment { \color[rgb]{0,0.5,0}\itshape  Кладем $A[j]$ в последовательность $A[1 \twodots j-1]$.}
	\li $i \gets j-1$
	\li \While $i > 0$ and $A[i] > \id{key}$
	\li \Do
		$A[i+1] \gets A[i]$
	\li $i \gets i-1$
	\End
	\li $A[i+1] \gets \id{key}$
		\End
\end{codebox}

\index{вставка в дерево}
\begin{codebox}
	\Procname{$\proc{\tt \textcolor{red}{Вставка в дерево}}(T,z)$}
	\li $y \gets \const{nil}$
	\li $x \gets \id{root}[T]$
	\li 
		\While $x \neq \const{nil}$
	\li 
			\Do
				$y \gets x$
	\li 
				\If $\id{key}[z] < \id{key}[x]$
	\li 			\Then $x \gets \id{left}[x]$
	\li 		\Else $x \gets \id{right}[x]$
				\End
			\End
	\li $p[z] \gets y$
	\li \If $y = \const{nil}$
	\li 	\Then
			$\id{root}[T] \gets z$\>\>\>\>\>\>\>\>\Comment { \color[rgb]{0,0.5,0}\itshape  Дерево было пусто }
	\li \Else
			\If $\id{key}[z] <\ id{key}[y]$
	\li 		\Then $\id{left}[y]\ gets z$
	\li 	\Else $\id{right}[y] \gets z$
			\End
		\End
\end{codebox}


\pagebreak
\subsection{algorithmic}

\subsubsection{С нумерацией строк}

\begin{algorithmic}[1]
	\FORALL{$i$ such that $0\leq i\leq 10$}
	\STATE carry out some processing
	\ENDFOR
\end{algorithmic}

\subsubsection{Большой пример}

\begin{algorithmic}
	\REQUIRE $n \geq 0$
	\ENSURE $y = x^n$
	\STATE $y \Leftarrow 1$
	\STATE $X \Leftarrow x$
	\STATE $N \Leftarrow n$
	\WHILE{$N \neq 0$}
		\IF{$N$ is even}
			\STATE $X \Leftarrow X \times X$
			\STATE $N \Leftarrow N / 2$
		\ELSE[$N$ is odd]
			\STATE $y \Leftarrow y \times X$
			\STATE $N \Leftarrow N - 1$
		\ENDIF
	\ENDWHILE
\end{algorithmic}

\subsubsection{Русский}

\realgorithmic

\begin{algorithmic}
	\REQUIRE $n \geq 0$
	\ENSURE $y = x^n$
	\STATE $y \Leftarrow 1$
	\STATE $X \Leftarrow x$
	\STATE $N \Leftarrow n$
	\WHILE{$N \neq 0$}
		\IF{$N$ is even}
			\STATE $X \Leftarrow X \times X$
			\STATE $N \Leftarrow N / 2$
		\ENDIF
	\ENDWHILE
\end{algorithmic}

\pagebreak %% Разрыв страницы :-)

        \section[Рисунки]{Растровая графика}
\subsection[Математика]{История математики, это 1 картинка}

\index{графика!pастровая}

	\begin{center} 
		\includegraphics[width=15cm]{img/math.jpg}
	\end{center}
	
\pagebreak %% Разрыв страницы :-)

\subsection{Пророчество}
	\begin{center} 
		\includegraphics[height=100mm]{img/theFutureofUsa.jpg}
	\end{center}
\subsection[Оси]{Оси и отрезки}
	\begin{center} 
		\includegraphics[width=6.3in]{img/l2-1-1.png}
	\end{center}
	

\pagebreak %% Разрыв страницы :-)

        \section[Векторная графика]{Векторная графика, tikz и  PSTricks}

\index{графика!векторная}

\subsection{tikz}

	\index{графика!векторная!tikz}
	
	\input{src/examples/plot-tikz}

\subsection{PSTricks}	
	\index{графика!векторная!PSTricks}
	
	\input{src/examples/plot-pstricks}
		
\pagebreak %% Разрыв страницы :-)


        \addtocontents{toc}{\protect\pagebreak}

        \supersection{Работа с текстом и шрифтами}
        \input{src/examples/text}

        \supersection{Для большой работы}
        
	\section{Введение}

Мы живем в мире информационных технологий, которые прочно вошли
в нашу жизнь. Мы пользуемся современными средствами связи.
Компьютер превратился в неотъемлемый элемент нашей жизни не только
на рабочем месте, но и в повседневной жизни. Быстрое развитие новых
информационных технологий свидетельствует о всевозрастающей роли
компьютерной техники в мировом информационном пространстве.

С каждым днем увеличивается число пользователей Интернета. Все
больше сетевые технологии оказывают влияние на развитие самой науки
и техники.



	% Введение
	%%%%%%%%%%%%%%%%%%%%%%%%%%%%%%%%%%%%%%%%%%%%%%%%%%%%%%%%%%%%%%%%%%%%%%%%%%%%%%%%
%%%
%%% Общие положения
%%%

\pagebreak

\section{Постановка}

\subsubsection{Дальнейшее развитие}

\subsubsection{Наивный подход}

\section{Заключение}
	    % Теоретическая часть
       % основная часть
        %%%%%%%%%%%%%%%%%%%%%%%%%%%%%%%%%%%%%%%%%%%%%%%%%%%%%%%%%%%%%%%%%%%%%%%%%%%%%%%%
%%%
%%% Фиктивная обложка диплома.
%%% На самом деле используется обложка из деканата
%%%

\begin{titlepage}

\newcommand{\byhand}[1]{\underline{\it \color{blue} \ #1\ }}

%%%% 
%%%% Фиктивная шапка. Похожа на настоящюю
%%%% 
{\small\begin{center}
		{\bfseries
			МИНИСТЕРСТВО ОБРАЗОВАНИЯ И НАУКИ РОССИЙСКОЙ ФЕДЕРАЦИИ
		} \\
		{ \footnotesize
			ФЕДЕРАЛЬНОЕ ГОСУДАРСТВЕННОЕ БЮДЖЕТНОЕ ОБРАЗОВАТЕЛЬНОЕ УЧРЕЖДЕНИЕ \\
					ВЫСШЕГО ПРОФЕССИОНАЛЬНОГО ОБРАЗОВАНИЯ \\
		}
		{\bfseries
			<<МОСКОВСКИЙ АВИАЦИОННЫЙ ИНСТИТУТ \\	
			(национальный исследовательский университет)>> (МАИ)\\
		}
		\begin{tabular}{p{13cm}}
			\hline \\
		\end{tabular}\\
		{Факультет №8\\
			{ \footnotesize  Прикладная математика и физика }
		}
\end{center}}

\vspace{24pt}

%%%% 
%%%% Помета о лицензии
%%%% 
{ \small \begin{flushright}
		\begin{tabular}{rl}
			Распространяется: & \byhand{на правах рукописи.} \\
		\end{tabular}
\end{flushright}}

\vspace{24pt}
	
%%%% 
%%%% Что
%%%% 
\begin{center}
	\sffamily
	{ \Large
		\begin{onehalfspacing}
			Шаблонный титульный лист для важных работ
		\end{onehalfspacing}
	}

    \vspace{24pt}

    { \large
        \begin{onehalfspacing}
            При содействии интернет-кинотеатра \href{http://tvzavr.ru}{TVzavr} \\
            \vspace{30pt}
            %\includegraphics[width=5cm]{./img/tvzavr.pdf}
        \end{onehalfspacing}
    }


\end{center}

\vspace{120pt}

%%%% 
%%%% Кто
%%%% 
{ \small \begin{flushright}
		\begin{tabular}{rl}
			%Руководитель работы: 		& \byhand{В.\,Н. Лукин}  \\
			Автор: 				& \byhand{И.\,К. Никитин}	\\
								&	\\
			Дата:				& \byhand{10 сентября 2012} 	\\
		\end{tabular}
\end{flushright}}

\vfill

%%%% 
%%%% Дата
%%%% 
{ \small \begin{center} %% ПО ЦЕНТРУ
		Москва~2012~г.
\end{center}}
	
\end{titlepage}
  % титульный лист (2)
        
\nocite{*}

\phantomsection
\renewcommand{\refname}{Список использованных источников}
\addcontentsline{toc}{section}{Список использованных источников}


\bibliographystyle{ugost2008ls}
\bibliography{src/graduate/biblio/main}

     % список использованных источников


        % лекции %%%%%%%%%%%%%%%%%%%%%%%%%%%%%%%%%%%%%%%%%%%%%%%%%%%%%%%%%%%%%%%%%%%

        %	\input{src/lct-01} %% лекция #1

        % лабы\курсовые %%%%%%%%%%%%%%%%%%%%%%%%%%%%%%%%%%%%%%%%%%%%%%%%%%%%%%%%%%%%%%%%%%%

        %	\input{src/work-problem} 		%% постановка
        %	\input{src/work-theory} 		%% теоретическая часть
        %	\input{src/work-solution} 		%% решение
        %	\input{src/work-example} 		%% примеры
        %	\input{src/work-conclusions} 	%% выводы

        % предметный указатель %%%%%%%%%%%%%%%%%%%%%%%%%%%%%%%%%%%%%%%%%%%%%%%%%%%%%%%%%%%%%%%%%%%
        %%%
        %%% дополнительное (свое) задание верхнего колонтитула для предметного указателя
        %%%

            %	\makeatletter
            %	\renewcommand{\@oddhead}{ \textcolor{blue}{Лекция (задача) \arabic{lections}} \hfil \par
            %	\hfil  \leftmark \hfil \rightmark }
            %	\makeatother

		\printindex
    \end{onehalfspacing}
\end{document}

%%
%%
%%
